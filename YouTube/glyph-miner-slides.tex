\documentclass{beamer}
\usepackage{graphicx}
\usepackage{caption}
\usepackage{appendixnumberbeamer}

\title{Glyph Miner - Video Summary}
\author{}
\date{}

\begin{document}

\frame{\titlepage}

% ----------------------
% Slides with Screenshots
% ----------------------

\begin{frame}{Introduction}
\includegraphics[width=0.8\textwidth]{frame01.png}
\caption*{This is Glyph Miner, a system for exploring and mining the structure and semantics of typefaces. It enables interactive visualization and analysis of glyphs — the individual characters in a font. The system is designed for tasks such as comparing glyphs, finding similarities across fonts, and creating new typeface designs.}
\end{frame}

\begin{frame}{Layout Extraction}
\includegraphics[width=0.8\textwidth]{frame02.png}
\caption*{We begin by selecting a font and loading its glyphs. Glyph Miner then extracts the layout of each glyph by tracing its contours and simplifying the geometry. This produces a graph-like structure where edges correspond to strokes or curves, and nodes indicate junctions or terminals.}
\end{frame}

\begin{frame}{Edit Layers}
\includegraphics[width=0.8\textwidth]{frame03.png}
\caption*{Layouts can be explored and edited interactively using layers. For example, we can highlight edges that form the main stems of a glyph. Other layers mark serifs, bowls, and other components. These layers are stored as annotations and can be turned on and off.}
\end{frame}

\begin{frame}{Glyph Comparison}
\includegraphics[width=0.8\textwidth]{frame04.png}
\caption*{Glyphs can be compared by loading multiple fonts side-by-side. Here, we visualize the letter “A” in five typefaces. The extracted layouts make it easy to spot differences in structure, such as stroke widths, proportions, and serif shapes.}
\end{frame}

\begin{frame}{Alignment and Normalization}
\includegraphics[width=0.8\textwidth]{frame05.png}
\caption*{To further support comparison, the layouts can be aligned and normalized. This makes structural similarities more apparent and helps with clustering or classification.}
\end{frame}

\begin{frame}{Statistical Measures}
\includegraphics[width=0.8\textwidth]{frame06.png}
\caption*{The system provides statistical measures of shape complexity, such as the number of edges, path length, and curvature. These can be visualized across fonts or plotted as histograms.}
\end{frame}

\begin{frame}{Generative Tasks}
\includegraphics[width=0.8\textwidth]{frame07.png}
\caption*{Glyph Miner can also be used for creative tasks. Here, we merge the layouts of two glyphs to generate a new design. The system interpolates between shapes and preserves structural features.}
\end{frame}

\begin{frame}{Export and Integration}
\includegraphics[width=0.8\textwidth]{frame08.png}
\caption*{Layouts can also be exported for use in other applications such as machine learning, typography research, or digital humanities. The system outputs SVG files, annotated graphs, and other formats.}
\end{frame}

\begin{frame}{Technical Details}
\includegraphics[width=0.8\textwidth]{frame09.png}
\caption*{Glyph Miner is implemented in Python and uses libraries such as OpenCV and NetworkX. It supports TrueType and OpenType fonts and provides a graphical interface for exploration and editing.}
\end{frame}

\begin{frame}{Summary}
\includegraphics[width=0.8\textwidth]{frame10.png}
\caption*{The system was developed as part of a research project on computational typography. It is available as open-source software and can be extended with custom layers or processing modules. Glyph Miner — an interactive system for mining typographic structures.}
\end{frame}

% ----------------------
% Transcript Appendix
% ----------------------

\appendix
\begin{frame}[plain]
  \frametitle{Appendix}
  \centering \Huge Full Transcript
\end{frame}

\begin{frame}[allowframebreaks]{Full Transcript}
\scriptsize
\begin{itemize}
\item This is Glyph Miner, a system for exploring and mining the structure and semantics of typefaces.
\item It enables interactive visualization and analysis of glyphs — the individual characters in a font.
\item The system is designed for tasks such as comparing glyphs, finding similarities across fonts, and creating new typeface designs.
\item We begin by selecting a font and loading its glyphs.
\item Glyph Miner then extracts the layout of each glyph by tracing its contours and simplifying the geometry.
\item This produces a graph-like structure where edges correspond to strokes or curves, and nodes indicate junctions or terminals.
\item Layouts can be explored and edited interactively using layers.
\item For example, we can highlight edges that form the main stems of a glyph.
\item Other layers mark serifs, bowls, and other components.
\item These layers are stored as annotations and can be turned on and off.
\item Glyphs can be compared by loading multiple fonts side-by-side.
\item Here, we visualize the letter “A” in five typefaces.
\item The extracted layouts make it easy to spot differences in structure, such as stroke widths, proportions, and serif shapes.
\item To further support comparison, the layouts can be aligned and normalized.
\item This makes structural similarities more apparent and helps with clustering or classification.
\item The system provides statistical measures of shape complexity, such as the number of edges, path length, and curvature.
\item These can be visualized across fonts or plotted as histograms.
\item Glyph Miner can also be used for creative tasks.
\item Here, we merge the layouts of two glyphs to generate a new design.
\item The system interpolates between shapes and preserves structural features.
\item Layouts can also be exported for use in other applications such as machine learning, typography research, or digital humanities.
\item The system outputs SVG files, annotated graphs, and other formats.
\item Glyph Miner is implemented in Python and uses libraries such as OpenCV and NetworkX.
\item It supports TrueType and OpenType fonts and provides a graphical interface for exploration and editing.
\item The system was developed as part of a research project on computational typography.
\item It is available as open-source software and can be extended with custom layers or processing modules.
\item Glyph Miner — an interactive system for mining typographic structures.
\end{itemize}
\end{frame}

\end{document}

%%% Local Variables:
%%% mode: LaTeX
%%% TeX-master: t
%%% End:
